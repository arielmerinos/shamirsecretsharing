\documentclass[letterpaper]{article}
\usepackage[utf8]{inputenc}
\usepackage[spanish, mexico]{babel}
\usepackage{amssymb, amsmath}
\usepackage{graphicx}
\usepackage[margin=1.5cm,
vmargin={1.5cm,0.7cm},
includefoot]{geometry}
\usepackage{amsthm}
\usepackage{dsfont}
\usepackage{mathtools}
\usepackage{graphicx}
\usepackage{algorithmic}
\usepackage[linesnumbered,ruled,vlined]{algorithm2e}
\begin{document}

\setlength{\unitlength}{1cm}
\thispagestyle{empty}
\begin{picture}(19,3)
\put(-0.5,1.2){\includegraphics[scale=.20]{img/unam1.png}}
\put(16,1){\includegraphics[scale=.29]{img/fciencias1.png}}
\end{picture}

\begin{center}
	\vspace{-114pt}
	\textbf{\large Proyecto Final}\\
	\textbf{ Semestre 2021-1}\\
	Prof. José Galaviz Casas\\
	Ayud. María Ximena Lezama \\
	\textbf{Modelado y programación}\\[0.15cm]
	Kevin Ariel Merino Peña\footnote{Número de cuenta 317031326} Armando Abraham Aquino Chapa\footnote{Número de cuenta 317058163}\\
	\today
\end{center}
\vspace{-10pt}
\rule{19cm}{0.3mm}
{\large \textbf{README}\\
{\large Para ejecutar el programa de la mejor manera, situarse en la carpeta \textbf{shamirsecretsharing} del proyecto y escribir el siguiente comando:
\begin{center}
	\textbf{python3 src/main.py -c /archivocifrar /archivoevaluaciones t n}
\end{center}
El programa cuenta con tres modalidades principales:
\begin{enumerate}
	\item -c. Para cifrar
	\item -d. Para descifrar
	\item -h Obtener ayuda
\end{enumerate}
\end{document}